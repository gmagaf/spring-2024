\documentclass[10pt]{article}
\usepackage{amssymb, amsthm, amsmath}
\usepackage{tikz-cd}
\usepackage[shortlabels]{enumitem}

\newcommand{\comp}{
  \mathbin{\circ}
}
\newcommand{\id}{
  \mathtt{id}
}
 
\newcommand{\N}{
  \mathbb{N}
}
\newcommand{\nat}{
  \text{nat}
}
\renewcommand{\exp}{
  \text{exp}
}
\newcommand{\ind}{
  \text{ind}
}

\newcommand{\refl}{
  \text{refl}
}
\newcommand{\sym}[1]{
  \text{sym} \; #1
}
\newcommand{\trans}[2]{
  \text{trans}(#1, #2)
}

% helevels
\newcommand{\iscontr}{
  \mathtt{iscontr \;}
}
\newcommand{\isaprop}{
  \mathtt{isaprop \;}
}
\newcommand{\isofhlevel}{
  \mathtt{isofhlevel \;}
}
\newcommand{\isweq}{
  \mathtt{isweq \;}
}
 \newcommand{\hfiber}{
  \mathtt{hfiber \;}
}

\newtheorem{exercise}{Exercise}

\title{Homotopy Type Theory 2023-2024\\
       Assignment 2
      }
\date{\today}
\author{George Magafossis \\
University of Amsterdam \\
UvA id: 14686996 \\
\texttt{george.magkafosis@student.uva.nl}}

\begin{document}

\maketitle

\begin{exercise}
Show that the definitions of proposition are equivalent.
That is, for all types $P$ in the universe, we need
to construct a term of type
\[
 (\isaprop P) \simeq (\prod x, y : P, x = y)
\]
\begin{proof}
We first require to create a function from proofs
of $\isaprop$ to proofs of $\prod x, y : P, x = y$.
So, we assume we have a term $\mathtt{pProp}
: \isaprop P$ and also that we have two terms
$x, y : P$. We need to show that they are equal,
i.e there exists a term of type $x = y$. From the
fact that $P$ is a prop we particularly get that
the type $x = y$ is contractible. So, inhabited.
We call this function
\[
 f : (\isaprop P) \to \prod x, y : P, x = y
\]

In order to create a homotopy inverse we introduce
the following lemma: If $H : \prod x, y : P, x = y$
holds and $P$ is inhabited (by say $p : P$), then
$P$ is contractible. The proof is easily given by
the pair $(p, H p)$, i.e $P$ is inhabited and all
terms are equal with the inhabitant that we present.
Now, we can create the inverse
\[
 g : \prod x, y : P, x = y \to (\isaprop P)
\]
Assume we have a proof $H$ of $\prod x, y : P, x = y$.
Then, in order to show that $P$ is a prop we show that
for any $x, y : P$ the type $x = y$ is contractible.
Since we have in our hypothesis an inhabitant of
$P$ (that is $x : P$), we can apply the previous
lemma to get that $P$ is contractible. By applying the
cumulativity of hlevels we get that it is also
a proposition.

Now, we have to show that $f, g$ are homotopy
inverses. If we assume that we have a term
$p : \isaprop P$, then $p, g (f p) : \isaprop P$.
Since for all $n$ the types $\isofhlevel n \; P$
are propositions the type $p = g (f p)$
is contractible. So, in particular inhabited.
Conversely, if we assume $H : \prod x, y : P, x = y$
then we need a term of $f (g H) = H$. Since
this is a function type we can apply the
function extensionality axiom. So, we need to show
that $(f (g H)) \; x \; y = H \; x \; y$ holds for all $x, y : H$.
By $H$ we can get $g H : \isaprop P$. Again, from
the cumulative hierarchy of hlevels we get that
$P$ is also a set. As a result any two proofs
of equality of $x, y : P$ are equal. That is,
there exists a term of type $(f (g H)) \; x \; y = H \; x \; y$.
This concludes the proof.
\end{proof}
\end{exercise}

\begin{exercise}
For any two propositions $P, Q$ we have that
homotopy equivalence is logicaly equivalent to their
logical equivalence. That is
\[
 (P \simeq Q) \iff (P \iff Q)
\]
\begin{proof}
In order to prove this we need to construct to
functions with signatures $(P \simeq Q) \to (P \iff Q)$
and $(P \iff Q) \to (P \simeq Q)$.

For the first
one we assume that the two propositions
are equivalent, i.e there exists a function
$f : P \to Q$ which is a equivalence. So, in
order to prove $P \iff Q$ we need two functions
with signatures $P \to Q$ and $Q \to P$. For
the first one we can provide exactly
$f : P \to Q$. For the second one, if we suppose
that we have a term $q : Q$ from the fact that
$f : P \to Q$ is an equivalence we can get that
its fiber is contractible. So, in particular
there exists a term $p : P$ such that $f p = q$.
We can choose $p$ to be the image of $q$.

For the second function $(P \iff Q) \to (P \simeq Q)$
we assume that we have $P \iff Q$. That means
that we get two functions $f : P \to Q$ and
$g : Q \to P$. We need to show that they are
homotopy inverses of each other. To that end, we
need $g (f p) = p$ for all $p : P$ and $f (g q) = q$
for all $q : Q$. But these types are contractible
by the assumption that $P, Q$ are propositions.
\end{proof}
\end{exercise}

\begin{exercise}
Show that the dependent product type former
commutes with $\isaprop$.
\begin{proof}
Assume that we have a type family $B$ depending
over a type $A$ and a dependent function
$p: \prod x : A, \isaprop (B x)$. We need to show
that the type $\prod x : A, (B x)$ is a proposition.
So, we assume that we have two dependent functions
$f, f' : \prod x : A, (B x)$ and we need to prove
that $f = f'$. We can apply function extensionality.
So, we need to show that $f x = f' x$ for any $x : A$.
But this follows from the fact that $B (x)$ is
a proposition, witnessed by $p x : \isaprop (B x)$.
\end{proof}
\end{exercise}

\begin{exercise}
Show that $\isweq f$ (is-contr f in Rijke) is a
proposition.
\begin{proof}
By the first exercise it suffices to show
that for any two proofs $s, t : \isweq f$,
we have $s = t$. A proof of $\isweq$
provides for any term $y : B$ in the codomain
of $f$, a proof that its type of
fibers is contractible. So, we can use
function extensionality and try to prove
that $s y = t y$. Since the type
$\iscontr (\hfiber f y)$ is the type of proofs
that a type has 0 hlevel, we get that it is a
proposition. So, the type $s y = t y$ is
contractible, hence inhabited.
\end{proof}
\end{exercise}

\begin{exercise}
Show that an equivalence between propositions
is (equivalent to) a logical equivalence.
\begin{proof}
Assume that we have two propositions $P, Q$.
From the second exercise we get that
\[
 (P \simeq Q) \iff (P \iff Q)
\]
That means that we get automatically two
functions $f : (P \simeq Q) \to (P \iff Q)$
and $g : (P \iff Q) \to (P \simeq Q)$. We
need to show that these are homotopy
inverses of each other.
We can show that the types $P \simeq Q$
and $P \iff Q$ are also propositions.
From that it follows that $g (f e) = e$
for all $e : (P \simeq Q)$ and that
$f (g e) = e$ for all $e : (P \iff Q)$.

So, we need to show that $P \simeq Q$
and $P \iff Q$ are propositions. For,
the first one by applying the first
exercise we need to show that for any
two $e, e' : P \simeq Q$ we have that
$e = e'$. Since $P \simeq Q$ is the sigma type
$\sum f : P \to Q, \isweq f$, and $\isweq f$
is a proposition (by previous exercise)
we get that we only need the underlying
functions of $e$ and $e'$ to be equal. By
previous exercise (dependent product commutes with $\isaprop$)
the function type $P \to Q$ is a proposition,
because $Q$ is a proposition. So, we get that
$e = e'$.

In order to prove that $P \iff Q$ is a proposition,
again we use the second exercise. That means that
we assume that we have two different terms of
$P \iff Q$ and we want to show that they are equal.
Since $P \iff Q$ is a product type, we can show
that the two terms are equal componentwise.
So, for any two logical equivalences $e, e' : P \iff Q$
we need to show that $\pi_1 e = \pi_1 e'$
and $\pi_2 e = \pi_2 e'$. Since, $\prod$
commutes with $\isaprop$ and both $P, Q$ are
propositions we have that the function types
$P \to Q$ and $Q \to P$ are also propositions.
So, we get the required equalities.

\end{proof}
\end{exercise}


\end{document}
