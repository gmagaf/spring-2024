\documentclass{article}

\usepackage{amsthm,amsfonts}
\usepackage{mathpartir}
\usepackage{mathtools}
\usepackage{tikz-cd}
\usepackage{hyperref,cleveref}
\usepackage{color,soul}
\usepackage{mathabx}
\usepackage{enumerate}

\theoremstyle{definition}
\newtheorem{definition}{Definition}[section]

% \theoremstyle{theorem}
\newtheorem{theorem}[definition]{Theorem}
\newtheorem{lemma}[definition]{Lemma}
\newtheorem{corollary}[definition]{Corollary}
\newtheorem{problem}[definition]{Problem}
\newtheorem{exercise}[definition]{Exercise}
\newtheorem{example}[definition]{Example}

\newenvironment{construction}{\begin{proof}[Construction]}{\end{proof}}



\newcommand{\types}{\mathcal T}
\newcommand{\type}{\ \textsc{type}}
\newcommand{\terms}{\mathcal S}
\newcommand{\contexts}{\mathcal C}
\newcommand{\context}{\textsc{ctx}}
\newcommand{\T}{\mathbb T}
\newcommand{\C}{\mathcal C}
\newcommand{\D}{\mathcal D}
\newcommand{\A}{\mathcal A}
\newcommand{\M}{\mathcal M}
\newcommand{\W}{\mathcal W}
\newcommand{\F}{\mathcal F}
\newcommand{\Set}{{\mathcal S}et}
\newcommand{\syncat}[1]{\C [#1]}
\newcommand{\defeq}{\coloneqq}
\newcommand{\interp}[1]{\lceil #1 \rceil}
\newcommand{\seq}{\doteq}
\newcommand{\lists}{\mathcal Lists}
\newcommand{\variables}{\mathcal Var}
\newcommand{\Epsilon}{\mathrm E}
\newcommand{\Zeta}{\mathrm Z}
\newcommand{\mor}{\mathrm {mor}}
\newcommand{\op}{\mathrm {op}}
\newcommand{\grpd}{\mathcal G}
\newcommand{\Id}{\mathtt {Id}}
\newcommand{\Path}{\mathrm {Path} \ }

\newcommand{\twist}{\mathrm{tw}}
\newcommand{\pr}{\mathrm{pr}}
\newcommand{\pair}[2]{\langle #1,#2\rangle}
\newcommand{\comp}{\mathbin{\circ}}
\newcommand{\sym}{\mathrm{sym}}
\newcommand{\conc}{\mathrm{conc}}
\newcommand{\idarrow}[1][]{\mathrm{id}_{#1}}
\newcommand{\opF}[1]{{#1}^{\mathrm{op}}}



\tikzset{pullbackcorner/.style={minimum size=1.2ex,path picture={
\draw[opacity=1,black,-,#1] (-0.5ex,-0.5ex) -- (0.5ex,-0.5ex) -- (0.5ex,0.5ex);%
}}}

\newcommand{\pullback}{\arrow[dr, phantom, "" {pullbackcorner} , very near start]}



\title{Homotopy Type Theory 2023-2024\\
       Assignment 4
      }
\date{\today}
\author{George Magafossis \\
University of Amsterdam \\
UvA id: 14686996 \\
\texttt{george.magkafosis@student.uva.nl}}

\begin{document}

\maketitle

\noindent\textbf{Problem 1.} Show the following.

\begin{lemma}
    \label{lem:equivalence relation}
    Nice path objects give reflexive and symmetric relations. \\
    If there are classes of morphisms $\mathcal L, \mathcal R$ such that
    \begin{enumerate}[(1)]
     \item each $r_A \in \mathcal L$
     \item each $\epsilon_A \in \mathcal R$
     \item $\mathcal R$ is stable under pullback and closed under composition
     \item $\mathcal L$ is stable under pullback along $\mathcal R$
     \item each morphism to the terminal object is in $\mathcal R$
     \item $\mathcal L \boxslash \mathcal R$
    \end{enumerate}
    then we get an equivalence relation.
\end{lemma}
\begin{proof}
To prove that the path object
$P A$ of an object $A$ is an equivalence
relation in $A$ we need to show that it
is reflexive, symmetric and transitive.

To show that it is reflexive we need
to prove that the diagonal map $A \to
A \times A$ factors through $PA$.
That means that there exists
a map $A \to PA$ such that the
following diagram commutes:
\[
 \begin{tikzcd}
  A \ar[rr, "\delta"] \ar[dr, dashed, "r_A"] & & A \times A \\
  & PA \ar[ur, "\epsilon_B"] &
 \end{tikzcd}
\]
This holds for $r_A$ by definition
of a path object.

To show that it is symmetric we need
to prove that the map
\[
 PA \xrightarrow{\epsilon_A} A \times A \xrightarrow{\twist} A \times A
\]
factors through $PA$, as before.
Here $\twist$ is given by $\pair{\pr_1}{\pr_0} : A \times A \to A \times A$.
We consider the following lifting problem
\[
 \begin{tikzcd}
  A \ar[rr, "r_A"] \dar{r_A} & & PA \dar["\epsilon_A"] \\
  PA \ar[urr, dashed, "\sym"] \rar["\epsilon_A"'] & A \times A \rar["\twist"'] & A \times A
 \end{tikzcd}
\]
This is indeed a lifting problem since
$\twist \comp \epsilon_A \comp r_A = \twist \comp \delta = \delta = \epsilon_A \comp r_A$. Since
$r_A \in \mathcal{L}$ from (1), $\epsilon_A \in \mathcal{R}$
from (2) and $\mathcal{L} \boxslash \mathcal{R}$
from (6) we get that there exists
a solution $\sym : PA \to PA$. The commutativity
of the lower triangle is what we wanted.

To show that it is transitive, we create
the object of all ``composable'' paths as the
following pullback. Here,
we will use $s \defeq \pr_0 \epsilon_A$
and $t \defeq \pr_1 \epsilon_A$.
\[
 \begin{tikzcd}
  PA \times_A PA \dar \rar \pullback & PA \dar["s"] \\
  PA \rar["t"] & A
 \end{tikzcd}
\]
To show that $PA$ is transitive we need to
prove that the arrow
\[
 \pair{s \comp \pr_0}{t \comp \pr_1} : PA \times_A PA \to A \times A
\]
factors through $PA$. To do so,
we form the following square:
\[
 \begin{tikzcd}[column sep=large]
  A \rar["r_A"] \dar["\pair{r_A}{r_A}"'] & PA \dar["\epsilon_A"] \\
  PA \times_A PA \rar["\pair{s \comp \pr_0}{t \comp \pr_1}"'] & A \times A
 \end{tikzcd}
\]
This commutes because
$\pair{s \comp \pr_0}{t \comp \pr_1} \comp \pair{r_A}{r_A} =
\pair{s \comp r_A}{t \comp r_A} = \epsilon_A \comp r_A$.
We know that $\epsilon_A \in \mathcal{R}$ from (2).
If we prove that $\pair{r_A}{r_A} \in \mathcal{L}$,
then we get the required factorization through
$PA$.


We will prove this in the following way:
We will show that
\begin{align*}
 \pair{r_A}{\idarrow[A]} : A \to PA {}_{t}\times_{\idarrow[A]} A \in \mathcal{L} \\
 \idarrow[PA] \times r_A : PA_t \times_{\idarrow[A]} A \to PA \times_A PA \in \mathcal{L}
\end{align*}
Then, we get $\pair{r_A}{r_A} = (\idarrow[PA] \times r_A) \comp \pair{r_A}{\idarrow[A]}$
\[\begin{tikzcd}
	A \\
	& {PA {}_t\times_{\idarrow[A]} A} \pullback & A \\
	& PA & A & {PA {}_t\times_s PA} \pullback & PA \\
	&&& PA & A
	\arrow["{\pair{r_A}{\idarrow[A]}}", from=1-1, to=2-2]
	\arrow["\idarrow", bend left=20, from=1-1, to=2-3]
	\arrow["{r_A}"', bend right=20, from=1-1, to=3-2]
	\arrow[from=2-2, to=2-3]
	\arrow[from=2-2, to=3-2]
	\arrow["\idarrow \times r_A", near end, bend left=10, from=2-2, to=3-4]
	\arrow["\idarrow", from=2-3, to=3-3]
	\arrow["{r_A}", bend left=20, from=2-3, to=3-5]
	\arrow["t", from=3-2, to=3-3]
	\arrow["\idarrow"', bend right=20, from=3-2, to=4-4]
	\arrow[from=3-4, to=3-5]
	\arrow[from=3-4, to=4-4]
	\arrow["s", from=3-5, to=4-5]
	\arrow["t"', from=4-4, to=4-5]
\end{tikzcd}\]
from the commutativity of the above diagrams.
Also, we can prove  that
$\pair{r_A}{r_A} \in \mathcal{L}$
since $\mathcal{L}$ is closed under composition.
Indeed, if $a, b$ are composable and in
$\mathcal{L}$ then for any $f \in \mathcal{R}$
we can find a lift to the following
problem:
\[\begin{tikzcd}
	\bullet & \bullet \\
	\bullet \\
	\bullet & \bullet
	\arrow[from=1-1, to=1-2]
	\arrow["a"', from=1-1, to=2-1]
	\arrow["f", from=1-2, to=3-2]
	\arrow[dashed, from=2-1, to=1-2]
	\arrow["b"', from=2-1, to=3-1]
	\arrow[dashed, from=3-1, to=1-2]
	\arrow["h"', from=3-1, to=3-2]
\end{tikzcd}\]
by first lifting $h \comp b$ and then
solving the lifting problem given by
the created lower square. The last
lift can be easily proven to be a solution
to the initial lifting problem.

To show that $\pair{r_A}{\idarrow[A]}$
is in $\mathcal{L}$ we will prove that
\[\begin{tikzcd}
	A & A \\
	{PA {}_t\times_{\idarrow[A]} A} & PA
	\arrow["\idarrow", from=1-1, to=1-2]
	\arrow["{\pair{r_A}{\idarrow}}"', from=1-1, to=2-1]
	\arrow["{r_A}", from=1-2, to=2-2]
	\arrow["\pr_0"', from=2-1, to=2-2]
\end{tikzcd}\]
is a pullback square. The square commutes
by definition and since the top arrow is
an identity, it has the universal property
of a pullback. $\pr_0$ is the pullback of
$\idarrow[A]$ along $t$. It is easy
to show that $\idarrow \in \mathcal{R}$, so
by (3) we get that $\pr_0 \in \mathcal{R}$.
From (4) we get that $\pair{r_A}{\idarrow} \in \mathcal{R}$.

We do the same for the square
\[\begin{tikzcd}
    X \ar[drr, bend left=15, "f"] \ar[ddr, bend right=15, "g"'] \ar[dr, dashed, "\pair{\pr_0 \comp g}{f}"]& & \\
	& {PA {}_t\times A} & A \\
	& {PA {}_t\times_{s} P A} & PA
	\arrow["\pr_1", from=2-2, to=2-3]
	\arrow["{\idarrow \times r_A}", from=2-2, to=3-2]
	\arrow["{r_A}", from=2-3, to=3-3]
	\arrow["\pr_1"', from=3-2, to=3-3]
\end{tikzcd}\]
Again, the square commutes by definition.
Suppose we have two arrows $f : X \to A$
and $g : X \to {PA {}_t\times_{s} P A}$
such that $r_A \comp f = \pr_1 \comp g$.
Then the arrow $\pair{\pr_0 \comp g}{f}$
makes the two triangles commute:
\begin{align*}
 & \pr_1 \comp \pair{\pr_0 \comp g}{f} = f \\
 & (\idarrow \times r_A) \comp \pair{\pr_0 \comp g}{f} = \pair{\pr_0 \comp g}{r_A \comp f} = \pair{\pr_0 \comp g}{\pr_1 \comp g} = g
\end{align*}
Additionally, for any other
$\pair{h_0}{h_1} : X \to PA {}_t\times A$
making the triangles commute we get
that
\begin{align*}
 & \pr_1 \comp \pair{h_0}{h_1} = f \implies h_1 = f \\
 & (\idarrow \times r_A) \comp \pair{h_0}{h_1} = g \implies h_0 = \pr_0 \comp g
\end{align*}
So, $\pair{\pr_0 \comp g}{f}$ is unique
with such property. Hence the square
is a pullback.

We have that $\pr_1$ is the pullback
of $t$ along $s$. Since $\mathcal{R}$
is closed under pullbacks by (3)
and every arrow to the terminal object is
in $\mathcal{R}$ by (5), we get that
Cartesian projections are in $\mathcal{R}$.
Since $\epsilon_A \in \mathcal{R}$
and $\mathcal{R}$ is closed under composition,
we get that $t \in \mathcal{R}$. So,
$\pr_1 \in \mathcal{R}$ from (3). Since
the above square is a pullback of $r_A \in \mathcal{L}$
along $\pr_1 \in \mathcal{R}$, from (4)
we get that $\idarrow \times r_A \in \mathcal{L}$
\end{proof}


\noindent\textbf{Problem 2.} Show the following.

\begin{lemma}
    \label{lem:factorization}
    Consider a weak factorization system $(\mathcal L, \mathcal R)$ on a category $\C$. We have that $\mathcal L$ is the class of all maps $f: X \to Y$ for which the following lifting problem has a solution.
    \[
         \begin{tikzcd}
             X \ar[r,"\lambda_f"] \ar[d,"f"] & Mf \ar[d,"\rho_f"]
             \\ 
             Y \ar[r,equal] & Y
         \end{tikzcd}
         \tag{$*$}
    \]

    Dually, $\mathcal R$ is the class of all maps $f: X \to Y$ for which the following lifting problem has a solution.
    \[
         \begin{tikzcd}
             X \ar[r,equal] \ar[d,"\lambda_f"] & X \ar[d,"f"]
             \\ 
             Mf \ar[r,"\rho_f"] & Y
         \end{tikzcd}
         \tag{$**$}
    \]
\end{lemma}
\begin{proof}
We need to show that
$\mathcal{L} = \{f : \begin{tikzcd}
                      X \ar[r,"\lambda_f"] \ar[d,"f"] & Mf \ar[d,"\rho_f"] \\
                      Y \ar[r,equal] & Y
                     \end{tikzcd}
 \text{ has a solution}\}$.
Suppose that we have $f \in \mathcal{L}$.
Then $f \boxslash \rho f$ since $\rho f \in \mathcal{R}$.
For the converse, suppose we have
$f :X \to Y$ such that the above square has a
solution:
\[
 \begin{tikzcd}
  X \ar[r,"\lambda_f"] \ar[d,"f"] & Mf \ar[d,"\rho_f"] \\
  Y \ar[r,equal] \ar[ur, dashed, "k"] & Y
\end{tikzcd}
\]
and we are given a lifting problem (solid diagram):
\[
 \begin{tikzcd}
  X \rar["\tau"] \dar["\lambda f"'] & Z \ar[dd, "g \in \mathcal{R}"] \\
  M f \dar["\rho f"'] \ar[ur, dashed, "h"] & \\
  Y \rar["\sigma"] \ar[uur, dashed, "h \comp k"'] & W
 \end{tikzcd}
\]
Since $\lambda f \in \mathcal{L}$
we can find a lift $h$ of $\sigma \comp \rho f$
against $g$. We can create a solution
$h \comp k : Y \to Z$. Indeed,
we have that
\begin{align*}
 g \comp h \comp k = \sigma \comp \rho f \comp k = \sigma \comp \idarrow[Y] = \sigma \\
 h \comp k \comp f = h \comp \lambda f = \tau
\end{align*}

We apply the same idea for proving
that
\[
 \mathcal{R} = \{f: X \to Y : \text{the diagram $**$ has a solution}\}
\]
First, we assume that we have a morphism
$f \in \mathcal{R}$. Since $\lambda f \in \mathcal{L}$
we get that $\lambda f \boxslash f$. Hence,
$**$ has a solution.

Now, assume that we have $f : X \to Y$
such that the diagram
\[
 \begin{tikzcd}
    X \ar[r,equal] \ar[d,"\lambda_f"] & X \ar[d,"f"] \\
    Mf \ar[r,"\rho_f"] \ar[ur, dashed, "k"] & Y
 \end{tikzcd}
\]
has a lifting $k : M f \to X$ such that
\begin{align*}
 k \comp \lambda f = \idarrow[X] \\
 f \comp k = \rho f
\end{align*}
In order to show that $f \in \mathcal{R}$
we will prove that it has the right lifting
property against any $m : A \to B \in \mathcal{L}$.
So, we assume that we have the following
lifting problem
\[
 \begin{tikzcd}
  A \ar[dd, "m"] \rar["\tau"] & X \dar["\lambda f"] \\
  & Mf \dar["\rho f"] \\
  B \rar["\sigma"] \ar[ur, dashed, "h"] & Y
 \end{tikzcd}
\]
Since $\rho f \in \mathcal{R}$ we get that
$m \boxslash \rho f$. So, there exists a lift
$h : B \to M f$ such that:
\begin{align*}
 \rho f \comp h = \sigma \\
 h \comp m = \lambda f \comp \tau
\end{align*}
As a result, we can form a lifting to the initial
problem. We observe that
\begin{align*}
 (k \comp h) \comp m = k \comp \lambda f \comp \tau = \tau \\
 f \comp (k \comp h) = \rho f \comp h = \sigma
\end{align*}
So, $k \comp h : B \to X$ is a solution.
Therefore, we can conclude that $f \in \mathcal{R}$.
\end{proof}


\newpage

\noindent\textbf{Problem 3.} In about 250 words, explain both the advantages that cubical type theory has over HoTT as well as the disadvantages.

We can give the following review as a brief comparison
between cubical type theories and HoTT. \\
Advantages:
\begin{enumerate}
 \item Cubical Type Theories enjoy canonicity. That means
 that we can prove the meta-theoretical fact that
 any term of the type of natural numbers given in
 the empty context is definitionally equal to
 zero or to the $m$-times application of the successor
 to zero for some meta-theoretic natural number $m$.
 This is not the case for HoTT.
 \item Cubical Type Theories are created by drawing
 inspiration from their models in Cubical Sets.
 These models are constructive. More precisely,
 we can construct using Intuitionistic Logic
 that CuTT has a model in cubical sets with
 identity types, sigma, Pi types and a univalent
 universe. This gives a computational meaning
 in the axioms of HoTT such as the Univalence
 axiom or function extensionality. On the contrary,
 the model we have for HoTT in simplicial sets
 is not constructive and so we cannot normalize
 such terms.
\end{enumerate}
Disadvantages:
\begin{enumerate}
 \item There is not one canonical choice for a
 category of cubes. This leads to various
 models in different versions of Cubical sets,
 like Cartesian, De Morgan and others.
 Unfortunately, the model structures in these
 models are not all Quillen equivalent to
 the category of spaces. Instead, HoTT
 is interpreted inside simplicial sets
 with the Kan model structure, which is
 Quillen equivalent to spaces.
 \item Also, not all cubical type theories
 are implemented in a proof assistant
 whereas there are extensive libraries of
 univalent foundations (HoTT) in Coq, Agda etc.
\end{enumerate}
Advantage or Disadvantage:
\begin{itemize}
 \item In Cubical type theories, the geometric
 features are more explicit rather than HoTT.
 In cubical type theories, judgments are usually
 indexed by a first layer context of ``dimension''
 variables, which is interpreted as a geometric
 cube, a representable object in the category
 of cubical sets. Additionally, one has to give
 account of how types behave under the geometric
 operations of cubes, in the form of composition
 rules. A similar idea is applied with
 glueing operations in the level of universes.
 Finally, paths in CuTT are closer to the
 intuitive idea we have for paths
 as functions of the interval into a type.
 All of the above geometric characteristics are
 either missing or expressed internally in HoTT.
\end{itemize}

\pagebreak

\noindent\textbf{Problem 4.} In about 250 words, explain why defining higher categories in HoTT has thus far not been successful, and give some examples of extensions of HoTT that have been proposed to ameliorate these problems.

We can give the following brief presentation 
regarding the problem of defining higher categories
in HoTT. There are many approaches to higher
categories in classical mathematics. Most of
them use set theory, for instance
defining $(\infty, 1)$-categories inside
simplicial sets. These approaches can be
internalized inside any proof-assistant that
can reason about set theory, and HoTT in
particular.

The issue arises when one tries to internalize
Reszk's definition of $(\infty, 1)$-categories
as simplicial spaces. We know that in HoTT,
types are interpreted as spaces. However,
these spaces are not organised in a 1-category,
like the category of spaces, but in an
$(\infty,1)$-category themselves. That means
that if we create a functor
$\opF{\Delta} \to \mathcal{U}$ it wouldn't be
strict and the functor laws would hold up to
higher homotopy. Furthermore, we are not
able to extend the idea of univalent categories
that we have for 1-categories. This works
for defining $(n,1)$-categories, but it would
require an infinite amount of data to
witness all the coherence conditions of an
$(\infty, 1)$-category.

The solutions proposed are extending HoTT
with various features that try to resolve this
issue. The idea is to replicate the notion
of Complete Segal Space, which is the version
of simplicial spaces in bisimplicial sets.
The first idea is to internalize a
type of strict equality. This leads to Two-Level
Type Theory which has two different equality
types, one internal for which Univalence axiom
holds and one external for which we have UIP.
We can import the notion of Complete Segal
Spaces in 2LTT. Another idea, comes from the
Riehl-Schulmann type theory. This is very similar
to cubical type theories, in which we index
our HoTT judgments by a layer of cubes, which
can be restricted to simplicies. With suitable
conditions, we can define Segal types and
Rezk types which can be interpreted as Segal
spaces and complete Segal spaces. Finally,
other approaches suggest altering the notion
of Identity types to directed equality types.
\end{document}
