\documentclass[10pt]{article}
\usepackage{amssymb, amsthm, amsmath}
\usepackage{tikz-cd}
\usepackage[shortlabels]{enumitem}

\newcommand{\comp}{
  \mathbin{\circ}
}
\newcommand{\id}{
  \mathtt{id}
}
 
\newcommand{\N}{
  \mathbb{N}
}
\newcommand{\nat}{
  \text{nat}
}
\renewcommand{\exp}{
  \text{exp}
}
\newcommand{\ind}{
  \text{ind}
}

\newcommand{\refl}{
  \text{refl}
}
\newcommand{\sym}[1]{
  \text{sym} \; #1
}
\newcommand{\trans}[2]{
  \text{trans}(#1, #2)
}

\newtheorem{exercise}{Exercise}

\title{Homotopy Type Theory 2023-2024\\
       Assignment 1
      }
\date{\today}
\author{George Magafossis \\
University of Amsterdam \\
UvA id: 14686996 \\
\texttt{george.magkafosis@student.uva.nl}}

\begin{document}

\maketitle

\begin{exercise}
Consider types $P, Q, R$ and terms $f : P \to Q$,
$g : Q \to R$ and $p : P$. Then we can construct
a term in $R$.
\begin{proof}
We can apply $f$ to $p$ to get a term $f \; p$ in $Q$
and then apply $g$ to that, to get a term $g (f \; p) : R$.
\end{proof}
\end{exercise}

\begin{exercise}
Consider types $P, Q, R$. Then
$(P \to (Q \times R)) \to (P \to Q)$.
\begin{proof}
In order to construct a function term of type
$(P \to (Q \times R)) \to (P \to Q)$ we will use
lambda abstraction twice. Suppose we have a term
$f : P \to Q \times R$ and a term $p : P$.
By applying $f$ to $p$ we get a term $qr : Q \times R$.
By projecting out the first component of $qr$
we get a term of type $Q$.
\end{proof}
\end{exercise}


\begin{exercise}
Define the exponential function
$\exp : \nat \to \nat \to \nat$.
\begin{proof}
We will define $\exp$ inductively on the
second argument: the exponent. So, assume that
we have $n, e : \nat$. By induction on $e$
we need to provide a term in $\nat$
for when $e = 0$. We choose $1 : \nat$
as the intended behavior is $\exp \; n \; 0 \doteq 1$.
Now, suppose we have a term $h : \nat$, the
value of the function $\exp \; n \; e$.
We need to provide a term in $\nat$ which should
correspond to $\exp \; n \; (S \; e)$. We give
$h * n$. So, by lambda abstractions we have defined:
\[
 \exp := \lambda n : \nat, \lambda e : \nat, \ind_\nat(1, [e', h].h*n) : \nat \to \nat \to \nat
\]
\end{proof}
\end{exercise}

\begin{exercise}
Given $l, m, n : nat$ we need to prove that
$\exp \; l \; (m + n) = (\exp \; l \; m) * (\exp \; l \; n)$.
\begin{proof}
We can prove this by induction on $n : \nat$.
If $n = 0$ we need to show
$\exp \; l \; (m + 0) = (\exp \; l \; m) * (\exp \; l \; 0)$.
By definition of $\exp$ this is the same as
$\exp \; l \; (m + 0) = (\exp \; l \; m) * 1$. We know that
$m + 0 = m$ and so, we can (by applying maponpaths)
simplify to $\exp \; l \; m = (\exp \; l \; m) * 1$.
Also, we know that $k * 1 = k$ for any $k : \nat$, so
we need to show $\exp \; l \; m = (\exp \; l \; m)$
which is trivially true.

Now, suppose that we have a proof $H$ of
$\exp \; l \; (m + n) = (\exp \; l \; m) * (\exp \; l \; n)$.
We need to prove that the same equation holds for
$S \; n$. Again, by definition of $\exp$ we have
$\exp \; l \; (S \; n) \doteq (\exp \; l \; n) * l$.
So, we will show two equalities and use the transitivity
of equality. First we will prove
\[
 \exp \; l \; (m + (S \; n)) = (\exp \; l \; (m + n)) * l
\]
and then
\[
 (\exp \; l \; (m + n)) * l = (\exp \; l \; m) * ((\exp \; l \; n) * l)
\]
The first equality holds as
\begin{align*}
 &\exp \; l \; (m + (S \; n)) = \text{ (by maponpaths and $m + (S \; n) = S(m + n)$)}\\
 &\exp \; l \; (S \; (m + n)) = \text{ (by defintion)} \\
 &(\exp \; l \; (m + n)) * l
\end{align*}
For the second equation we use $H$ to show that
\[
 (\exp \; l \; (m + n)) * l = ((\exp \; l \; m) * (\exp \; l \; n)) * l
\]
and then by associativity of the multiplication
on the natural numbers we get
\[
 (\exp \; l \; (m + n)) * l = (\exp \; l \; m) * ((\exp \; l \; n) * l)
\]
\end{proof}
\end{exercise}

\begin{exercise}
Suppose we have two paths (proofs of identity type)
in a type $A$: $p : a = b, q: a' = b$. We should construct
a term $path\_combination \; p \; q: a = a'$.
\begin{proof}
First we can prove $b = a'$. This follows from the
fact that equality is symmetric. So, we have
$\sym{q} : b = a'$. Then, by transitivity
we can get $\trans{p}{\sym{q}} : a = a'$.
\end{proof}
\end{exercise}

\begin{exercise}
Assume we have a path $p : a = b$ between two
terms of type $A$. Then, $\refl_a = path\_combination \; p \; p$
\begin{proof}
We can prove this by using path induction on $p$.
So, we need to find a proof that
\[
 \refl_a = \trans{\refl_a}{\sym{\refl_a}}
\]
This is definitionaly true by the way we define
$\sym{(-)}$ and $\trans{-}{-}$.
\end{proof}
\end{exercise}

\end{document}
