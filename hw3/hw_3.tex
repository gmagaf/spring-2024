\documentclass[10pt]{article}
\usepackage{amssymb, amsthm, amsmath}
\usepackage{tikz-cd}
\usepackage[shortlabels]{enumitem}

\newcommand{\comp}{
  \mathbin{\circ}
}
\newcommand{\id}{
  \mathtt{id}
}
 
\newcommand{\N}{
  \mathbb{N}
}
\newcommand{\nat}{
  \text{nat}
}
\renewcommand{\exp}{
  \text{exp}
}
\newcommand{\ind}{
  \text{ind}
}

\newcommand{\refl}{
  \text{refl}
}
\newcommand{\sym}[1]{
  \text{sym} \; #1
}
\newcommand{\trans}[2]{
  \text{trans}(#1, #2)
}

\newtheorem{exercise}{Exercise}

\title{Homotopy Type Theory 2023-2024\\
       Assignment 3
      }
\date{\today}
\author{George Magafossis \\
University of Amsterdam \\
UvA id: 14686996 \\
\texttt{george.magkafosis@student.uva.nl}}

\theoremstyle{definition}
\newtheorem*{definition*}{Definition}

\newcommand{\mor}{\text{mor}}
\newcommand{\D}{\mathcal{D}}
\newcommand{\I}{\mathbb{I}}
\renewcommand{\comp}{\mathbin{\circ}}

% Diagram
\newsavebox{\pullback}
\sbox\pullback{
\begin{tikzpicture}
\draw (0,0) -- (1ex,0ex);
\draw (1ex,0ex) -- (1ex,1ex);
\end{tikzpicture}}

\begin{document}

\maketitle

\begin{exercise}
Consider the category $G$ of groupoids, and the class
$\D \subseteq \mor(G)$ of isofibrations.
Show that this is a display map category, and that it
has the two additional properties required of a display
map category. That is, show that:
\begin{enumerate}[i)]
 \item every $X \to \star$ is a display map
 \item $\D$ is closed under isomorphisms
 \item $\D$ is stable under pullback
 \item $\D$ contains all isomorphisms
 \item $\D$ is closed under composition
\end{enumerate}

\begin{proof}
We have the following definition of isofibrations:
\begin{definition*}
An isofibration is a functor $p : E \to B$
such that for any object $x$ in $E$ and any
isomorphism $g : p(x) \to y$ in $B$, there exists
an iso $f : x \to x'$ in $E$ such that $p(f) = g$.
\end{definition*}
This is equivalent to the following: For any
solid commutative square
\[
 \begin{tikzcd}
  \star \dar["0"'] \rar["x"] & E \dar["p"] \\
  \I \rar["g"] \ar[ur, dashed, "f"] & B
 \end{tikzcd}
\]
there exists a lift $f : \I \to E$ making
the two triangles commute. Here, $\I$ is
the free iso groupoid, i.e a category with
two objects and an iso between them.

\begin{enumerate}[i)]
 \item Suppose that we have the following
 commutative square:
 \[
  \begin{tikzcd}
   \star \dar["0"'] \rar["x"] & X \dar["!"] \\
   \I \rar["g"] & \star
  \end{tikzcd}
 \]
 Then, from the universal property of $\star$
 we know that $g = !$, i.e it is the unique
 functor sending all objects to the only object
 of $\star$ and all morphisms to the identity
 morphism in $\star$. So, we can lift $g$ to
 the identity morphism of $x$ in $E$.
 This is an iso with domain $x$ and gets
 mapped by $! : X \to \star$ to the identity
 morphism in $\star$. So, it makes both triangles
 commute.
 
 \item Suppose we have the following situation,
 where the square commutes and $\phi : E' \to E$,
 $\psi : B' \to B$ are isos of groupoids.
 \[
  \begin{tikzcd}
   E' \dar["p'"'] \rar["\phi"] & E \dar["p"] \\
   B' \rar["\psi"] & B
  \end{tikzcd}
 \]
 We need to show that if $p$ is an isofibration,
 then so is $p'$. For that we assume that we have
 a lifting problem as before (left square):
 \[
  \begin{tikzcd}
   \star \dar["0"'] \rar["x"] & E' \dar[very near start, "p'"'] \rar["\phi"] & E \dar["p"] \\
   \I \rar["g"'] \ar[urr, dashed, near start, "f"] & B' \rar["\psi"'] & B
  \end{tikzcd}
 \]
 Then, the outer square defines a lifting problem
 for $\psi(g)$. Since functors preserve isos,
 $\psi(g)$ is an iso. From the fact that $p$
 is an isofibration, there exists a lift
 $f : \I \to E$ such that the domain of $f$
 is $\phi(x)$ and $p(f) = \psi(g)$. The morphism
 $\phi^{-1}(f)$ is again an iso in $E'$. Its
 domain is $\phi^{-1}(\phi(x)) = x$ and
 it gets mapped by $p'$ to
 \[
  p'(\phi^{-1}(f)) = \psi^{-1}(p(f)) = \psi^{-1}(\psi(g)) = g
 \]
 That means that $\phi^{-1}(f)$ lifts $g$. So,
 $p'$ is an isofibration.
 
 \item Suppose that we have the following
 pullback square
 \[
  \begin{tikzcd}
   Q \dar["q"] \rar["H"] \arrow[dr, "\usebox\pullback", phantom, very near start, color=black] & E \dar["p"] \\
   C \rar["F"] & B
  \end{tikzcd}
 \]
 where $p$ is an isofibration. We need to show
 that $q$ is also an isofibration.
 Suppose that we have the following lifting problem
 (left square):
 \[
  \begin{tikzcd}
   \star \dar["0"'] \rar["x"] & Q \dar["q"'] \rar["H"] \arrow[dr, "\usebox\pullback", phantom, very near start, color=black] & E \dar["p"] \\
   \I \rar["g"'] \ar[urr, dashed, bend right=10, near start, "f"] \ar[ur, dashed, near start, "k"] & C \rar["F"] & B
  \end{tikzcd}
 \]
 From the fact that $p$ is an isofibration,
 there exists an iso $f : H(x) \to e$
 such that $p(f) = F(g)$. From the universal
 property of the pullback we get that there
 exists a unique functor $k : \I \to Q$ such
 that $H \comp k = f$ and $q \comp k = g$.
 We claim that $k$ is the required lifting.
 Since $k$ gets mapped to $g$ by $q$ we only
 need to verify that the domain of $k$ is $x$.
 This follows again from the universal property
 of the pullback. We can show that the domain
 of $H(k)$ is the domain of $f$, i.e $H(x)$,
 and the domain of $q(k)$ is the domain of $g$.
 From the uniqueness of $x : \star \to Q$
 we have that $x = 0 \comp k : \star \to Q$.
 This means that the domain of $k$ is $x$.
 
 \item Suppose that we have an iso $p : E \to B$.
 Then, for any object $x$ in $E$ and any
 isomorphism $g : p(x) \to y$, the morphism
 $p^{-1}(g) : x \to p^{-1}(y)$, which is an
 iso, gets mapped to $g$ by $p$.
 
 \item Suppose that $p : E \to B$ and
 $q : C \to E$ are two composable isofibrations
 and that we have the following lifting problem
 (solid square):
 \[
  \begin{tikzcd}
   \star \rar["x"] \ar[dd, "0"'] & C \dar["q"] \\
    & E \dar["p"] \\
   \I \ar[ur, dashed, "f"] \rar["g"] & B
  \end{tikzcd}
 \]
 From the fact that $p$ is an isofibration,
 there exists a lift $f : \I \to E$ such
 that $p(f) = g$ with domain $q(x)$.
 This defines a new lifting problem:
 \[
  \begin{tikzcd}
   \star \rar["x"] \dar["0"'] & C \dar["q"] \\
   \I \rar["f"] \ar[ur, dashed, "h"] & E
  \end{tikzcd}
 \]
 which has a solution $h : \I \to C$ such that
 $q(h) = f$, with domain $x$. This is a solution
 to the initial problem since $p(q(h)) = p(f) = g$.
 So, the composition $p \comp q$ is an
 isofibration as well.
\end{enumerate}

\end{proof}

\end{exercise}


\end{document}
